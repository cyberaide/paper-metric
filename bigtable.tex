
\begin{table*}[p]
\caption{Metrics Table}

\begin{scriptsize}
\label{T:metrics-bigtable}
\bigskip
\begin{center}
\begin{tabular}{lp{0.1\textwidth}p{0.1\textwidth}p{0.3\textwidth}p{0.3\textwidth}p{0.1\textwidth}}
ID & Name & Stackeholder & Description & Motivation & Sample \\
\hline
\rowcolor{blue!20} \multicolumn{6}{l}{\bf User related metrics} \\
\hline
UC.1&
User Count & 
~&
This metric counts the active users for the cloud &
User count is important for measuring the popularity of the cloud & 
~ \\
\hline
UC.2&
Major Users &
~&
This metric shows heavy users in terms of consuming resources e.g. top 10 users &
Top 10 Users is important to see who has the most impact on the shared resources & 
~ \\
\hline
UC.3 &
User Type &
~&
This metric shows a user type defined in an account system with a percentage of
the total numbers. e.g. a project leader, an instructor, or a students. & 
User Type is important to see a proportion of users. &
~ \\
\hline
UC.4 &
Repeat User&
~&
This metric shows the number of users who actively using services for a certain period.  e.g. last 3/6/9 month: 16/32/24 &
Repeat User is important for measuring user activity for a certain period &
~ \\
\rowcolor{blue!20} \multicolumn{6}{l}{\bf Virtual Machine related metrics} \\
\hline
VC.1&
VM Count & 
~&
This metric counts the launched VM instances on the cloud &
VM Count is important for measuring the volume of requested instances &
~ \\
\hline
VC.2&
VM Count by Project, Leader, or Institution &
~&
This metric shows share of resource by group metrics such as project, leader or institution &
Group usage is important for measuring group usage &
~ \\
\hline
VC.3 &
Current running VMs or accessed users &
~&
This metric shows instant usage data to see current status &
real time usage is important for checking peak or detecting unusual usage &
~ \\
\hline
VCT.1 &
VM Creation Time &
~&
This metric shows the latency of creating a new instance to ensure fast creation &
VM Create Time is important for measuring the latency &
~ \\
& & & & & \\
\hline
\rowcolor{blue!20} \multicolumn{6}{l}{\bf Image related metrics} \\
\hline
IC.1 &
Image Type &
~ &
This metric shows an image type registered in the cloud platforms with a
percentage of the total number. e.g. ubuntu14.04, centOS 7, or Fedora 22 & 
Image Type is important to see a proportion of images. & 
~\\
\hline
\rowcolor{blue!20} \multicolumn{6}{l}{\bf Flavor related metrics} \\
\hline
FC.1 &
Flavor Type &
~ &
This metric shows an flavor (instance) type registered in the cloud platforms with a
percentage of the total number. e.g. m1.small, m1.medium, or m1.xlarge & 
Flavor Type is important to see a proportion of instance types. & 
~\\
\hline
\rowcolor{blue!20} \multicolumn{6}{l}{\bf Storage related metrics} \\
\hline
SU.1 &
Disk Usage &
~&
This metric shows the size of Disks allocated to instances &
Disk Usage is important for measuring system resource used &
~ \\
\hline
SU.2 &
Object Usage &
~&
This metric shows the size of Object Storage allocated to instances &
Object Usage is important for measuring system resource used &
~ \\
\hline
SU.3 &
Block Usage &
~&
This metric shows the size of storage blocks mounted to instances &
Block Usage is important for measuring system resource used &
~ \\
\hline
\rowcolor{blue!20} \multicolumn{6}{l}{\bf Region related metrics} \\
\hline
RC.1 & 
Location & 
~& 
This metric shows a geographical location of a user. & 
Location is important for resource availability to different regions. & 
~\\
\hline
\rowcolor{blue!20} \multicolumn{6}{l}{\bf Server related metrics} \\
\hline
SC.1 & 
Node Distribution & 
~ &
This metric shows a physical node distribution. & 
Node Distribution is important for load banacing. & 
~ \\
\hline
\rowcolor{blue!20} \multicolumn{6}{l}{\bf Network related metrics} \\
\hline
NL.1 &
Latency &
~&
This metric shows a network and application performance &
Latency is important with acceptable and strict latency expectation &
~ \\
\hline
NT.2 &
Network Throughput&
~&
This metric shows the actual amount of data delivered successfully &
Network thoughput is important for measuring network performance &
~ \\
\hline
NP.3 &
PublicIP Count &
~&
This metric shows availability of public IP addresses. &
PublicIP Count is important for the number of available and free IP addresses. &
~ \\
\hline
NP.3 &
PriveIP Count &
~& 
This metric shows availability of private IP addresses. &
PrivateIP Count is important for the number of available and free IP addresses. &
~ \\
\hline
\end{tabular}
\end{center}
\end{scriptsize}
\end{table*}

\begin{table*}[p]
\caption{Metrics Table}
\begin{scriptsize}
\label{T:metrics}
\bigskip
\begin{center}
\begin{tabular}{lp{0.1\textwidth}p{0.1\textwidth}p{0.3\textwidth}p{0.3\textwidth}p{0.1\textwidth}}
\hline
ID & Name & Stackeholder & Description & Motivation & Sample \\
\hline
 \rowcolor{blue!20} \multicolumn{6}{l}{\bf Unordered metrics} \\
\hline
 & & & & & \\
\hline
RS&
Runtime Sum&
~&
This metric shows the total amount of runtime for launched instances &
runtime by hour is important for measuring actual runtime of instances &
~ \\
\hline
XX.X &
vCPU Usage &
~&
This metric shows the number of vCPU cores allocated to instances &
vCPU Usage is important for measuring system resource used &
~ \\
\hline
XX.X &
Memory Usage &
~&
This metric shows the size of Memories allocated to instances &
Memory Usage is important for measuring system resource used &
~ \\
\hline
XX.X &
Node Distribution&
~&
This metric shows a proper balancing of physical compute nodes &
Node Distribution is import for system load balance &
~ \\
\hline
XX.X &
Tenant Distribution (Std deviation)&
~&
This metric shows a proper balancing of resources per tenant. Useful to identify heavy tenants &
Tenant Distribution is important to see users spread evenly across the nodes &
~ \\
\hline
XX.X &
Availability&
~&
This metric shows a percentage rate of available resources to accept a new request &
Availability is important to provide cloud resources continuously
~ \\
\hline
XX.X &
Scalability \& Capacity&
~&
This metric shows an actual limit of a service or physical system in the cloud &
Scalability and Capacity are important to measure IaaS performance &
~ \\
\hline
XX.X &
Power Consumption&
~&
This metric shows the amount of energy used in the cloud platform  &
Electricity is important for measuring actual cost by Kilo Watt per hour (KWh) &
~ \\
\hline
XX.X &
Throughput &
~&
This metric shows the performance of cloud services by measuring completed tasks, i.e. PaaS &
Throughput is important for measuring service performance e.g. PaaS &
~ \\
\hline
XX.X &
CPU Speed (system performance)&
~&
This metric shows the performance of cloud resources by clock speed of a processor &
CPU Clock speed is important to understand an actual speed of CPUs over different cloud platforms &
~ \\
\hline
XX.X &
Memory Speed (system performance)&
~&
This metric shows the performance of cloud resources by clock speed of a memory &
Memory Clock speed is important to understand an actual speed of memories over different cloud platforms &
~ \\
\hline
XX.X &
Disk Speed (system performance)&
~&
This metric shows the performance of cloud resources by read/write speed of a disk including SSD &
Disk speed is important to understand an actual speed of disks over different disk types&
~ \\
\hline
\rowcolor{blue!20} \multicolumn{6}{l}{\bf Project metrics} \\
\hline
 & & & & & \\
XX.X & Number of projects & ~ & x description & time periods: daily, monthly, qaurterly, yearly & ~ \\ \hline
XX.X & Number of projects by discipline & ~ & x description & time periods: daily, monthly, qaurterly, yearly & ~ \\ \hline
XX.X & Number of projects by organization & ~ & x description & time periods: daily, monthly, qaurterly, yearly & ~ \\ \hline
XX.X & Count Technology & ~ & x description & time periods: monthly, qaurterly, yearly & ~ \\ \hline
XX.X & Count Desired Technology & ~ & x description & x motivation &~\\ \hline
XX.X & Comparision of desired Technologies & ~ & period: weekly, monthly,
yearly. Type: Table, Frequency, Pie chart  &~\\ \hline
XX.X & Histogram desired X over Period & ~ & x description & x motivation &~\\ \hline
\hline
\end{tabular}
\end{center}
\end{scriptsize}
\end{table*}
